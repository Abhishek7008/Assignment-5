\documentclass[journal,12pt,twocolumn]{IEEEtran}

\usepackage{setspace}
\usepackage{gensymb}

\singlespacing


\usepackage[cmex10]{amsmath}

\usepackage{amsthm}

\usepackage{mathrsfs}
\usepackage{txfonts}
\usepackage{stfloats}
\usepackage{bm}
\usepackage{cite}
\usepackage{cases}
\usepackage{subfig}

\usepackage{longtable}
\usepackage{multirow}

\usepackage{enumitem}
\usepackage{mathtools}
\usepackage{steinmetz}
\usepackage{tikz}
\usepackage{circuitikz}
\usepackage{verbatim}
\usepackage{tfrupee}
\usepackage[breaklinks=true]{hyperref}
\usepackage{graphicx}
\usepackage{tkz-euclide}
\usepackage{float}

\usetikzlibrary{calc,math}
\usepackage{listings}
\usepackage{color} %%
\usepackage{array} %%
\usepackage{longtable} %%
\usepackage{calc} %%
\usepackage{multirow} %%
\usepackage{hhline} %%
\usepackage{ifthen} %%
\usepackage{lscape}
\usepackage{multicol}
\usepackage{chngcntr}

\DeclareMathOperator*{\Res}{Res}

\renewcommand\thesection{\arabic{section}}
\renewcommand\thesubsection{\thesection.\arabic{subsection}}
\renewcommand\thesubsubsection{\thesubsection.\arabic{subsubsection}}

\renewcommand\thesectiondis{\arabic{section}}
\renewcommand\thesubsectiondis{\thesectiondis.\arabic{subsection}}
\renewcommand\thesubsubsectiondis{\thesubsectiondis.\arabic{subsubsection}}


\hyphenation{op-tical net-works semi-conduc-tor}
\def\inputGnumericTable{} %%

\lstset{
%language=C,
frame=single,
breaklines=true,
columns=fullflexible
}
\begin{document}


\newtheorem{theorem}{Theorem}[section]
\newtheorem{problem}{Problem}
\newtheorem{proposition}{Proposition}[section]
\newtheorem{lemma}{Lemma}[section]
\newtheorem{corollary}[theorem]{Corollary}
\newtheorem{example}{Example}[section]
\newtheorem{definition}[problem]{Definition}

\newcommand{\BEQA}{\begin{eqnarray}}
\newcommand{\EEQA}{\end{eqnarray}}
\newcommand{\define}{\stackrel{\triangle}{=}}
\bibliographystyle{IEEEtran}
\providecommand{\mbf}{\mathbf}
\providecommand{\pr}[1]{\ensuremath{\Pr\left(#1\right)}}
\providecommand{\qfunc}[1]{\ensuremath{Q\left(#1\right)}}
\providecommand{\sbrak}[1]{\ensuremath{{}\left[#1\right]}}
\providecommand{\lsbrak}[1]{\ensuremath{{}\left[#1\right.}}
\providecommand{\rsbrak}[1]{\ensuremath{{}\left.#1\right]}}
\providecommand{\brak}[1]{\ensuremath{\left(#1\right)}}
\providecommand{\lbrak}[1]{\ensuremath{\left(#1\right.}}
\providecommand{\rbrak}[1]{\ensuremath{\left.#1\right)}}
\providecommand{\cbrak}[1]{\ensuremath{\left\{#1\right\}}}
\providecommand{\lcbrak}[1]{\ensuremath{\left\{#1\right.}}
\providecommand{\rcbrak}[1]{\ensuremath{\left.#1\right\}}}
\theoremstyle{remark}
\newtheorem{rem}{Remark}
\newcommand{\sgn}{\mathop{\mathrm{sgn}}}
\providecommand{\abs}[1]{\left\vert#1\right\vert}
\providecommand{\res}[1]{\Res\displaylimits_{#1}}
\providecommand{\norm}[1]{\left\lVert#1\right\rVert}
%\providecommand{\norm}[1]{\lVert#1\rVert}
\providecommand{\mtx}[1]{\mathbf{#1}}
\providecommand{\mean}[1]{E\left[ #1 \right]}
\providecommand{\fourier}{\overset{\mathcal{F}}{ \rightleftharpoons}}
%\providecommand{\hilbert}{\overset{\mathcal{H}}{ \rightleftharpoons}}
\providecommand{\system}{\overset{\mathcal{H}}{ \longleftrightarrow}}
%\newcommand{\solution}[2]{\textbf{Solution:}{#1}}
\newcommand{\solution}{\noindent \textbf{Solution: }}
\newcommand{\cosec}{\,\text{cosec}\,}
\providecommand{\dec}[2]{\ensuremath{\overset{#1}{\underset{#2}{\gtrless}}}}
\newcommand{\myvec}[1]{\ensuremath{\begin{pmatrix}#1\end{pmatrix}}}
\newcommand{\mydet}[1]{\ensuremath{\begin{vmatrix}#1\end{vmatrix}}}
\numberwithin{equation}{subsection}
\makeatletter
\@addtoreset{figure}{problem}
\makeatother
\let\StandardTheFigure\thefigure
\let\vec\mathbf
\renewcommand{\thefigure}{\theproblem}
\def\putbox#1#2#3{\makebox[0in][l]{\makebox[#1][l]{}\raisebox{\baselineskip}[0in][0in]{\raisebox{#2}[0in][0in]{#3}}}}
\def\rightbox#1{\makebox[0in][r]{#1}}
\def\centbox#1{\makebox[0in]{#1}}
\def\topbox#1{\raisebox{-\baselineskip}[0in][0in]{#1}}
\def\midbox#1{\raisebox{-0.5\baselineskip}[0in][0in]{#1}}
\vspace{3cm}
\title{Assignment No.5}
\author{Abhishek Nayak}
\maketitle
\newpage
\bigskip
\renewcommand{\thefigure}{\theenumi}
\renewcommand{\thetable}{\theenumi}
Download all python codes from
\begin{lstlisting}
https://github.com/Abhishek7008/Assignment-5.git
\end{lstlisting}
%
and latex-tikz codes from
%
\begin{lstlisting}
https://github.com/Abhishek7008/Assignment-5.git
\end{lstlisting}
%
\section{Linear Forms Exercise 2.5(a)}
Find the coordinates of the foci, the vertices,
the lengths of major and minor axes and the
eccentricity of the ellipse.

\begin{lemma}
For ellipse,
Property:
\begin{align}
    \abs{\vec{V}} > 0
    \\
    \lambda_1>0,\lambda_2<0
\end{align}
Standard Form:
\begin{align}
    \frac{\vec{x}^T\vec{D}\vec{x}}{\vec{u}^T\vec{V}^{-1}\vec{u}-f}=1 \label{eq1}
\end{align}
Centre:
\begin{align}
    \vec{c} = -\vec{V}^{-1}\vec{u}
\end{align}
Axes:
\begin{align}
\begin{cases}
    \sqrt{\frac{\vec{u}^T\vec{V}^{-1}\vec{u}-f}{\lambda_1}}
    \\
    \sqrt{\frac{\vec{u}^T\vec{V}^{-1}\vec{u}-f}{\lambda_2}}
\end{cases}
\end{align}
Focus:
\begin{align}
    \vec{F} &= \sqrt{\frac{(\vec{u}^T\vec{V}^{-1}\vec{u}-f)(\lambda_2-\lambda_1)}{\lambda_1\lambda_2}}
\end{align}
\end{lemma}
\section{Solution}
The given matrix equation for ellipse
\begin{align}
   \vec{X}^T\myvec{9&0\\0&9}=36
\end{align} 
It can be represent as
   \begin{align}
    \vec{x}^T\myvec{\frac{1}{4}&0\\0&\frac{1}{9}}=1\label{1}
   \end{align}
   From the general formula of ellipse we get 
   \begin{align}
     \vec{u}^T\vec{V}^{-1}\vec{u}=1  
   \end{align}
   From equation \ref{1}, we can define the value of
   \begin{align}
        \lambda_1= \frac{1}{4} \\
        \lambda_2=\frac{1}{9}
   \end{align}
   So the major axis of ellipse is on Y-axis.\\\
   Focus 
   \begin{align}
       F= \myvec{0\\ \pm c}
       \end{align}\\
       \begin{align}
           F= \sqrt{\frac{(\vec{u}^T\vec{V}^{-1}\vec{u}-f)(\lambda_2-\lambda_1)}{\lambda_2\lambda_1}}
       \end{align}\\
       \begin{align}
           F= \sqrt{\frac{(\vec{u}^T\vec{V}^{-1}\vec{u}-f)(\lambda_2-\lambda_1)}{\lambda_2\lambda_1}}
       \end{align}
       \\
       \begin{align}
            F=\sqrt{\frac{\lambda_2-\lambda_1}{\lambda_2\lambda_1}}
       \end{align}
      \\
      \begin{align}
          F=\sqrt{\frac{\frac{1}{9}-\frac{1}{4}}{\frac{1}{9}\times\frac{1}{4}}}=\sqrt{5}
      \end{align}
       \\
       \begin{align}
           F=\myvec{0\\\pm\sqrt{5}}
       \end{align}
    
   Axis
   \begin{align}
       \sqrt{\frac{\vec{u}^T\vec{V}^{-1}\vec{u}-f}{\lambda_1}}=\sqrt{\frac{1}{\frac{1}{4}}}=2\\
       \sqrt{\frac{\vec{u}^T\vec{V}^{-1}\vec{u}}{\lambda_2}}=\sqrt{\frac{1}{\frac{1}{9}}}=3
   \end{align}
   Length of the Major Axis
   \begin{align}
       2\times3=6
   \end{align}
   Length of the Minor Axis
   \begin{align}
       2\times2=4
   \end{align}
   
  Vertices,
   \myvec{0\\ \pm 3}\\
   Eccentisity of Major Y-axis
   \begin{align}
       = C\sqrt{\lambda_2}\\
       =\sqrt{5}\times\frac{1}{3}\\
       ={\frac{{\sqrt{5}}}{3}}
   \end{align}
  
\end{document}